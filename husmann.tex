% !TeX root = RJwrapper.tex
\title{Multi-Purpose Optimization with flexible Simulated-Annealing}
\author{by Kai Husmann, Alexander Lange (and Elmar Spiegel)}

\maketitle

\abstract{
An abstract of less than 150 words.
}

\section{Introduction}

As early optimization methods occurred with the first digital computers \citep{Corana.1987}, nowadays several optimization methods for various purposes are available \citep{Wegener.2005}. According to the definition by \citet{Pronzato.1984}, optimization is the finding of an optimum with respect to one or more parameters and optional restrictions. Optimization procedures can be separated into two groups depending on their field of application. In case the loss function shows relatively simple responses, long-established and well-known exact optimization methods, e. g. the Linear Simplex-Method \citep{Dantzig.1959}, are usually the most efficient way to solve optimization problems. If all assumptions on loss and model restrictions are met, these methods will obligatorily find the exact solution. If, however, one of the assumptions (e. g. linearity or unimodality) is violated, more general optimization procedures must be applied to solve problems properly.  



This section may contain a figure such as Figure~\ref{figure:rlogo}.

\begin{figure}[htbp]
  \centering
  \includegraphics{Fig/Rlogo}
  \caption{The logo of R.}
  \label{figure:rlogo}
\end{figure}

\section{Another section}

There will likely be several sections, perhaps including code snippets, such as:

\begin{example}
  x <- 1:10
  result <- myFunction(x)
\end{example}

\section{Summary}

This file is only a basic article template. For full details of \emph{The R Journal} style and information on how to prepare your article for submission, see the \href{http://journal.r-project.org/share/author-guide.pdf}{Instructions for Authors}.

\bibliography{husmann}

\address{Author One\\
  Affiliation\\
  Address\\
  Country\\}
\email{author1@work}

\address{Author Two\\
  Affiliation\\
  Address\\
  Country\\}
\email{author2@work}

\address{Author Three\\
  Affiliation\\
  Address\\
  Country\\}
\email{author3@work}

% !TeX root = RJwrapper.tex
\title{Multi-Purpose Optimization with flexible Simulated-Annealing}
\author{by Kai Husmann, Alexander Lange (and Elmar Spiegel)}

\maketitle

\abstract{
An abstract of less than 150 words.
}

\section{Introduction}

As early computer-based optimization methods developed contemporaneously with the first digital computers \citep{Corana.1987}, nowadays numerous optimization methods for various purposes are available \citep{Wegener.2005}. One of the main challenges in Operations Research is thus to match the optimization problem with a reasonable method. According to  \citet{Kirkpatrick.1983}, optimization procedures in general can be distinguished into exact methods and heuristics. In case the loss function of the optimization problem shows quite simple responses, exact methods are often meaningful tools of choice. If all assumptions on model loss and restrictions are met, these methods will obligatorily find the exact solution. 
Exact methods are advantageous since they do not need much parameterizations. The  Linear Simplex-Method \citep{Dantzig.1959} as an example only needs the loss-function and optional restrictions as model input. If, however, any of the model assumptions, e. g. linearity or unimodality, is violated, exact methods are unable to solve the problems properly. Practically, they thus lack applicability whenever a loss function is too complex. With developing computer power 
Heuristics

, more general heuristics must be applied to solve problems properly. While heuristics can be parameterized

the definition by \citet{Pronzato.1984}, optimization is the finding of an optimum with respect to a vector of parameters.

This section may contain a figure such as Figure~\ref{figure:rlogo}.

\begin{figure}[htbp]
  \centering
  \includegraphics{Fig/Rlogo}
  \caption{The logo of R.}
  \label{figure:rlogo}
\end{figure}

\section{Another section}

There will likely be several sections, perhaps including code snippets, such as:

\begin{example}
  x <- 1:10
  result <- myFunction(x)
\end{example}

\section{Summary}

This file is only a basic article template. For full details of \emph{The R Journal} style and information on how to prepare your article for submission, see the \href{http://journal.r-project.org/share/author-guide.pdf}{Instructions for Authors}.

\bibliography{husmann}

\address{Author One\\
  Affiliation\\
  Address\\
  Country\\}
\email{author1@work}

\address{Author Two\\
  Affiliation\\
  Address\\
  Country\\}
\email{author2@work}

\address{Author Three\\
  Affiliation\\
  Address\\
  Country\\}
\email{author3@work}
